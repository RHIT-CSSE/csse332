%% -*- Mode: LATEX -*-

\documentclass[12pt]{article}

\usepackage{xcolor}
\usepackage[colorlinks]{hyperref}
\hypersetup{citecolor=pink}
\hypersetup{linkcolor=red}
\usepackage{cleveref}

\usepackage{multicol}
\usepackage{multirow}

\usepackage{graphicx}
\usepackage{longtable}
\usepackage[font=small,labelfont=bf]{caption} 


\newcommand{\figuresPath}[1]{../figures/#1}

\input{csse332_defs}
\input{csse332_schedule_dates.tex}
\input{csse332_labs_defs.tex}

\title{Lab 3 --- \labtitlethree}

\begin{document}

\maketitle

This and the remaining labs should be done in pairs -- only
\textbf{one} lab should be submitted.


\section{Objectives}

Following completion of this lab you should be able to:
\latex{\vspace*{-\topsep}\vspace*{-\partopsep}}
\begin{itemize}{}
  \latex{\setlength{\itemsep}{0in}}
  \item Create and use a Makefile to compile code.  
  \item Use the GNU Debugger, gdb, to debug your code.
  \item Implement a queue data structure in C.
  \item Use the \texttt{fork} system call to spawn child processes.
  \item Use the \texttt{exec} family of system calls to overlay the
  process image of a child process with another executable program.
  \item Use the \texttt{gettimeofday} system call to generate timing
  statistics for the creation and execution of processes.
  \item Explain process creation, process termination and process
  management.
\end{itemize}

\section{Tasks}

\subsection{Makefile}

Checkout the \texttt{maketutorial} project in the Examples directory
from your repository onto your local machine. Follow the tutorial.  A
solutions is posted for your convenience.

\subsection{GNU Debugger}

Follow the instructions in the \href{../resources/gdbtutorial.html}{gdb
tutorial} to familiarize yourself with using gdb to debug your code.  
gdb is a convenient utility that you should use to diagnose and correct 
bugs in your code.

\subsection{CharsStringsFiles (15 points)}

Checkout the \texttt{CharsStringsFiles} project from your repository
repository onto your local machine.  Type your answers to the 
questions below in the file \texttt{ANSWERS.txt}.


The \texttt{CharsStringsFiles} project is inspired by many high-level
string processing methods provided by Python's String class.  These
methods include \texttt{capitalize}, \texttt{lower}, \texttt{upper},
\texttt{swapcase}, \texttt{capwords}, \texttt{strip}, and
\texttt{center}, to name a few.  In order to provide similar
functionality in C, we designed and implemented this project.

We have provided all of the test code in the file \texttt{csf-main.c}.
We have also provided support for reading data from files and
displaying data to the console.

The following questions all refer to the code in the
\texttt{CharsStringsFiles} project.  The code consists of three C
source files and two header files as follows.

\begin{center}
  \begin{longtable}{p{4cm}p{10cm}}
    \texttt{file-functions.c} \newline \texttt{file-functions.h} &
      Implements file reading functions.  Three
      different versions of \texttt{printFileToConsole} are provided.
      The first reads a file and prints it to the console one character
      at a time.  The second reads and prints a line at a time.  The
      final version reads the entire file into an array, then prints
      the array. \\


    \texttt{string-functions.c} \newline  \texttt{string-functions.h} &
    Implements a number of  string functions such as
      \texttt{capitalize} and \texttt{center}. \\


    \texttt{csf-main.c} &
    Implements the sample application of chars,
      strings and files.  Sample input is provided in the
      \textbf{Data} folder and expected output is captured in
      \texttt{EXPECTED\_OUTPUT.txt}.\\
  \end{longtable}
\end{center}

\begin{enumerate}
  \item Your instructor will complete with you a makefile to compile 
  your code for this assignment.

  \item \label{q:error1} (5 points) %
  \texttt{csf-main} does not work as intended.  In fact, in it's
  current state, \texttt{csf-main} generates an error when it is run.
  Use gdb to identify which error is received and where.
  \begin{enumerate}
    \item What error is received?
    \vfill%

    \item Where (what line number and which instruction) is the error
    received?
    \vfill%
  \end{enumerate}
  \item \label{q:error1cause} (5 points) %
  What is the root cause of the error in Problem \ref{q:error1}?

  Hint: Use \texttt{gdb}.
  \vfill%


  \item (5 points) %
  Make the necessary modifications to correct the root cause of the
  error you identified in Problems \ref{q:error1} and
  \ref{q:error1cause}.  Be sure to commit your changes to your repo.

  Hint: You shouldn't need to understand the whole codebase to make
  the change - the solution only requires adding one line.  But make
  sure you at least understand the function you're editing.\\[\baselineskip]

\end{enumerate}

\subsection{Queue (100 points)}

In the exercise, you will use the notion of a linked-list to implement
a queue data structure.  Queues are heavily used in operating systems
for various tasks.  You will use pointers to create and maintain the
queue.  The functions to be implemented are described in the header
file included in the project that you will checkout from your
repository.
 
\textbf{Linked-list:} A linked list is a data structure that contains
a list of nodes.  A node has two parts:
\begin{enumerate}
  \item Data 
  \item A pointer to next node
\end{enumerate}

Typically, the last node in the list will not point to any other node;
hence, its next node pointer will point to \texttt{NULL}. Using the
next node pointer, it is possible to keep track of the entire
linked-list and access any node in the list.  A linked-list typically
has a head pointer, as shown in the figure below.  The head pointer
points to the first node in the list. The figure below shows a
singly-linked-list.
		
% \latex{\includegraphics{sll.png}}
% \html{\htmladdimg{sll.png}}

\includegraphics{\figuresPath{sll.png}}
                                
\textbf{Using a singly-linked-list to implement a queue:} A queue is a
data structure in which a node is always added to the \textbf{tail} of
the list and removed from the \textbf{head} of the list.  It is a data
structure with a \textit{First-In First-Out (FIFO)} discipline.  To
model this behavior, implement a singly-linked-list with a
\textbf{head} and a \textbf{tail} pointer.  The \textbf{tail} pointer
points to the last node in the list or \texttt{NULL} if the list is
empty.  The \textbf{head} pointer points to the first node in the list
or \texttt{NULL} if the list is empty.  Each node in the list will
have a pointer to the next node in the list or \texttt{NULL} if there
is no next node.
	  
The structure for a queue instance, the structure for a node, function
prototypes and their descriptions are available in \texttt{queue.h}.
	  
\begin{enumerate}
  \item Checkout the \textbf{Queue} project in the \textbf{lab3}
  directory from your repository onto your local machine.
	    
  \item In \texttt{queue.c}, implement the functions described in
  \texttt{queue.h}.  You must use pointers as described in the header
  file and you must use a singly-linked-list with head and tail
  pointers.  You are provided with \texttt{driver.c}
  to test your implementation.  You may \textbf{not}
  modify \texttt{queue.h}.
	    
  \item Create a \textbf{Makefile} for the Queue project that
  will build your executable.

  \item Commit your work, including the Makefile (but excluding your 
  executables, \texttt{*.o} files, and other vestigial files), to your 
  repository with an appropriate commit message.
\end{enumerate}
                                
Sample output for the printQueue function is provided below:

\begin{tabular}{cccc}
  Process id & Arrival Time & Service Time & Remaining Time \\ \hline
  1 & 0 & 10 & 10 \\
  2 & 1 & 11 & 11 \\
  3 & 2 & 12 & 12 \\
  4 & 3 & 13 & 13 \\
  5 & 4 & 14 & 14 \\ \hline
\end{tabular}

\subsection{createProcesses}

Checkout the \textbf{createProcesses} project in the \textbf{lab3}
directory from your repository onto your local machine.  The
files in this project should be used as sample files for this lab.
The lecture slides on \href{../slides/09-Processes.pptx}
{Process operations} may also be helpful.

\subsection{Processes (100 points)}

There are two different ways to organize a parent-children process
hierarchy.  The first is a process chain, in which the parent waits
for each child process to complete before creating a new child.  The
second is a process fan, which we will be using for in lab.  In a
process fan the parent creates all of the child processes at once
allowing them to execute in parallel.  The figures below show a process
fan, on the left, and a process chain, on the right.

\begin{center}
  \begin{tabular}{cc}
\includegraphics[width=3in]{\figuresPath{process_fan.png}} &
\includegraphics{\figuresPath{process_chain.png}} \\
a) Process fan & b) Process chain  \\
\end{tabular}
\captionof{figure}{Spawning child processes a): Process fan, b) Process chain}
\end{center}

Checkout the \textbf{Processes} project in the \textbf{lab3} directory
from your repository onto your local machine.

\begin{enumerate}
  \item Write a C program in the files \texttt{processes.c} and
  \texttt{processes.h} that creates child processes in a fan:
  \begin{enumerate}
    \item Accept three command line arguments: 1) $n$ the number of
    child processes to create; 2) $x$ a delay factor in microseconds
    for each child process; and 3) the name of a source file
    containing individual file names, one per line, to pass to the
    child processes.

    \texttt{Files.txt} included in the project is an example.
					 
    \item Use a loop to create a process fan of $n$ child processes
    that will run in parallel. 

    Recall: In a process fan, the children processes are all created
    by the parent process at once.  You need to enforce this fact by
    making sure that no child process forks.
					
    \item Each child process should be assigned a local id named
    \textbf{index}, which corresponds to the order in which it was
    created.  The first child process will have an \textbf{index} of
    1, second will have an \textbf{index} of 2, and
    so on.
					
    \item Each process, including the parent process, will also have a
    start time and an end time.
					
    \item The process \textbf{pid} (the id returned to the parent from
    \textbf{fork()}), process start time and process end time for each
    child process should be stored in an appropriate structure.
					
    \item An array of such structures should be created to store
    process info for the $n$ child processes.  The entry for each child
    process will be indexed by the process's \textbf{index}.
					
    \item Immediately before each child process is created, the parent
    should capture the current time and store it in the process info
    structure as the start time for that child process.  The process
    \textbf{pid} should also be stored in the structure.  The process
    start time should be captured using the \textbf{gettimeofday(2)}
    system call (``man -S 2 gettimeofday'' describes its
    usage).
					
    \item Each child process should start by sleeping
    (($n$-\textbf{index})*$x$) microseconds.  Use a value of 100,000
    microseconds for $x$ initially. This will be varied later.
					
    \item In each child process, read the corresponding line in the
    file passed into the program (e.g. \texttt{Files.txt}).  For
    example, the first process created should read the first line in
    the input file.  If $n$ is larger than the number of lines present
    in the file, the program should wrap back around to the first line
    in the input file, using the same file more than once.

    \item During execution and before terminating each child process
    should do the following:
    \begin{enumerate}
      \item Sleep (($n$-\textbf{index})*$x$) microseconds.

      \item Print its \textbf{index} to the console.

      \item Overlay its process image with the ``\texttt{myCopy}''
      program using one of the \href
      {http://pubs.opengroup.org/onlinepubs/000095399/functions/exec.html}
      {exec family of system calls}.
      Remember that ``\texttt{myCopy}'' takes two command line
      arguments: 1) the name (or path) of the input file (read from
      the source file by the child process); and 2) the name of the
      corresponding output file.  If the input file name is
      \texttt{filename.txt}, then the output file name should be
      \texttt{filename\_index\_out.txt}.  For example,
      \texttt{FileOne.txt} will correspond with
      \texttt{FileOne\_1\_out.txt} if the child process with index 1
      produces the output file.
    \end{enumerate}

    \item After a child terminates, the parent should print the
    child's \textbf{index} and its total time from creation to end of
    execution to the console in the following form:
\begin{verbatim}
Time for process <index> = <calculated time> microseconds.
\end{verbatim}

    Note that the parent must get the termination time of the
    terminated process and the appropriate entry from the array of
    process info structures to compute the total time for that
    process.  The end time for the terminated process should also be
    stored in its process info structure.
					
    \item The parent process should measure its total execution time
    using \textbf{gettimeofday(2)}.  After all of the children
    have finished execution, the parent should print to the console
    that the children have finished, that the current process is the
    parent process, and the measured total execution time before
    it exits.  Ultimately, a parent's total execution time should be
    less-than or equal to the sum of the execution times for all its
    child processes.  Can you think of why that might be?
  \end{enumerate}

  \item Once your program works with the fan, experiment with
  different $x$ values for the sleep times of each process.  By
  looking at the program output, you should see when processes begin to
  call \textbf{myCopy} simultaneously and when they begin to reorder themselves
  again.

  \item When you turn in your assignment it should include
  the following:
  \begin{enumerate}
    \item Makefile and source code both \texttt{processes.c} and
    \texttt{myCopy.c}. 

    Make sure that your Makefile creates \textbf{both} executables.

    \item A README file containing usage instructions.

    \item Console printout results for at least three test cases with
    different values of $n$ and $x$.

    \item The input files you used for the \textbf{processes} and
    \textbf{myCopy} executables.
  \end{enumerate}
\end{enumerate}

\section{Linting your code before you make your final submission}

We will use the \href{http://google.github.io/styleguide/cppguide.html}{Google C++
  style}.  Use
  \href{../resources/cpplint.py}{\textbf{cpplint.py}}
  (this version has a couple of small local modifications to make it
  work better for C code) to make sure your code conforms to a
  reasonable style.  You should download \texttt{cpplint.py} to your
  home directory (accesible via \texttt{cd \textasciitilde{}}).  You might need
  to make \texttt{cpplint.py} executable (e.g. by running
  ``\texttt{chmod a+x cpplint.py}''). You can check \texttt{foo.c} by
  running ``\texttt{\textasciitilde{}/cpplint.py hw01.c}'' assuming
  \texttt{cppylint.py} is in your home directory.  If
  \texttt{cpplint.py} is in a different directory (which is not
  recommended), you'll need to include the path to it.  If it's in the
  current directory the path to it is ``\texttt{./cppylint.py}.''
  Note: you should place
  \href{../resources/CPPLINT.cfg}{CPPLINT.cfg} in your
  home directory (preferred option) OR the root of your CSSE332
  directory tree to automatically load local configuration options.
  This config file will eliminate the copyright warning.

  A simple way to download these files to your home directory is by 
  entering the commands below at the terminal:

\begin{verbatim}
> cd ~
> wget http://classes.csse.rose-hulman.edu/csse332/resources/cpplint.py
> wget http://classes.csse.rose-hulman.edu/csse332/resources/CPPLINT.cfg
> chmod a+x cpplint.py
\end{verbatim}

\section{Turning It In}

This lab is due no later than 11:59 PM on the due date.

You will receive credit for each project \textbf{only} if it compiles
and runs on the course linux distribution.

\begin{enumerate}
  \item Only \textbf{one} lab should be submitted by each
  \textbf{pair}.  You \textbf{must} include both partner's names in a
  comment at the top of all files you submit.  Commit a
  ``\texttt{README}'' file in the repository of the partner not
  submitting the code which lists the names of both partners.

  \item Add any new files/directories you created to your repository.

  \item Commit and push your solutions to each project to your repository 
  with appropriate commit messages.

  \item Use the git status command and/or point your web browser to
  your repository to verify that you submitted your solutions.
\end{enumerate}

Note: It is good practice to commit often.

Once the labs are graded, you can do a \textit{git pull} to pull 
comments and grader feedback.

\end{document}

% Local Variables:
% mode:latex
% End:
