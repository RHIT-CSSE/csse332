\documentclass[11pt,usletter]{article}

% some packages
\usepackage{xspace}
\usepackage{amsmath,amssymb}
\usepackage{graphicx}
\usepackage{listings}
\usepackage[x11names]{xcolor}

\usepackage[%
  top=1in,%
  bottom=1in,%
  left=1in,%
  right=1in]{geometry}

% set title
\title{%
  {\huge CSSE332: \textsc{Operating Systems}}\\%
  \vspace{2em}
  Buffalo, Mark, and Mohammad's Guide to Condition Variables}
\date{Last updated: \today}

%------------------------------------------------------------------------------
%   LSTLISTINGS SETTINGS
%------------------------------------------------------------------------------
\definecolor{codegreen}{rgb}{0,0.6,0}
\definecolor{codegray}{rgb}{0.5,0.5,0.5}
\definecolor{codepurple}{rgb}{0.58,0,0.82}
\definecolor{backcolour}{rgb}{0.95,0.95,0.92}

% define our own style for the netsec labs
\lstdefinestyle{netseclab}{%
  backgroundcolor=\color{backcolour},
  basicstyle=\tt\footnotesize,
  tabsize=2,
  numbers=left,
  stringstyle=\color{codepurple},
  keywordstyle=\color{blue},
  numberstyle=\footnotesize\color{codegray},
  commentstyle=\color{codegreen},
  numbersep=4pt,
  breaklines=true,
  keepspaces=true,
  showstringspaces=false,
  upquote=true,
  frame=single,
}

\begin{document}

\maketitle

\section{Introduction}

Most of the simple solutions to concurrency problems using condition variables
fall into a simple style.  This is not necessarily the only way condition
variables can be applied correctly but doing it this way makes it easier to see
if things are correct.  Also limiting your options to one style makes it less
overwhelming to start until things start to feel natural and obvious.  So I
recommend you follow this system.

{\bf Note that this guidance is rules of thumb and hints}; it’s not a
substitute for really thinking about your problem.  You won’t get leverage with
the graders by saying {\em ``hey this incorrect thing I did is consistent with
the guidance''} or {\em ``the guidance didn’t warn me not to incorrect thing I
did''}.  The goal is to solve the problem without concurrency bugs – any style
advice that introduces bugs should be ignored.

Although we can provide hints, you still must figure out your concurrency
state, decide when threads need to wait, decide when threads need to signal,
etc. That creative problem solving is the essence of writing correct concurrent
code, and learning that is the essential skill we hope you will internalize on
Exam 2.  That kind of problem solving can never be condensed into a simple
recipe, that's why some day you’ll hopefully get paid the \$\$.

\section{Getting Started}

Here are some general guidelines to get you started:

\begin{itemize}
  \item You'll need one mutex lock, any number of condition variable depending
    on the number of reasons to wait.

    \noindent Note that the mutex is to protect the state of the world (or the
    concurrency state) and not the critical section. So any time you touch the
    state of the world, make sure you hold the lock on you!

  \item You'll also need some number of variables to represent the {\em
    concurrency state} (aka, the state of the world). I use concurrency state
    here to represent part of the (usually global) state that your code
    accesses to enforce correctness.

    \noindent For example, if there are only three threads allowed in the
    critical section, we need some variable to represent how many threads are
    currently in that critical section.

  \item I often have some idea of what the condition variables/concurrency
    state will be when I start, but end up tweaking them as I design the
    solution and write the code.
\end{itemize}

\section{In Your Code}

Here are some considerations to keep in mind when writing code (or pseudocode):

\begin{itemize}
  \item If you thread might wait {\bf or} read/update the concurrency state,
    lock the mutex!

  \item When holding the lock, you can safely read/write at a consistent time.
    Usually, you'll want to hold the lock for the entire read/write experience.
    In other words, there should be only one mutex and that its critical
    section should be as large as possible; it doesn’t matter how fast your
    multithreaded code is, because it doesn’t count if it isn’t correct.

    \noindent For example, don't lock/unlock to read/write variable {\tt A},
    then lock/unlock to read/write variable {\tt B}, as this will tend to
    introduce bugs where {\tt A} and {\tt B} have an inconsistent state.
    Similarly, don't read variable {\tt A}, lock, unlock, re-lock, and then
    write variable {\tt A}.

    Here is a more concrete example:

    \begin{lstlisting}[style=netseclab,language=c]
void *thread_fn(void *ignore) {
  pthread_mutex_lock(&lock);
  while(people_in_room > 5) {
    pthread_cond_wait(&cv, &lock);
  }
  pthread_mutex_unlock(&lock);

  // this is dangerous, this gap in time where you don't hold the lock
  // might allow another thread to sneak by and change the value of
  // people_in_room.
  // Remember that the scheduler hates you! If a context switch is going
  // to hurt you, then that's when the scheduler will do it!

  pthread_mutex_lock(&lock);
  people_in_room++;
  pthread_mutex_unlock(&lock);

  // enter room now!
  ...
}
    \end{lstlisting}

  \item If reading the concurrency state indicates you need to wait, then use
    wait on an appropriate condition variable {\bf within a while loop} (and
    this implicitly unlocks).

  \item  After you wake from a wait, remember you are holding the lock again so
    it is safe to read or write to the concurrency state.

  \item  If it is safe to proceed (as opposed to waiting) unlock the mutex
    after you know it’s safe and after you have updated the state appropriately
    (see example above).

  \item  Never hold the mutex for a long amount of time (more accurately, I
    usually say never hold the mutex for an unbounded amount of time).
    Especially never hold the mutex during one of the sleeps we use to simulate
    long processing in these problems. 

    \noindent Holding the mutex for a long time makes it unsafe to read/update
    the concurrency state (because doing so might introduce a long pause or
    deadlock).

  \item  Never use the mutex to protect something – only use the mutex to
    safeguard the concurrency state.  Use waiting to ensure that all threads do
    not proceed when it is unsafe to do so. 

  \item  {\bf Every wait needs a corresponding signal or signals} (at least one
    signal or broadcast).  Consider carefully whether you wish to use signal or
    broadcast. Insert the signal/broadcasts where they are needed. 

    \noindent Try to avoid threads waking up and immediately re-sleeping but
    sometimes this is unavoidable. 

  \item  {\bf Never busy wait}.  A process that is unable to proceed because
    it’s waiting for some other process to work should always be waiting for a
    signal.

  \item Always grab the lock before signaling. It might seem superfluous but it
    will help you avoid very weird conditions that are dependent on the
    implementation of the {\tt pthreads} library.

  \item  Do not make any assumptions about the order in which thing wake up on
    a {\tt pthread\_cond\_signal}. From the {\tt POSIX} specification, we know
    that at least one thread will wake up, but which one is left for the
    scheduler/implementation to determine.
\end{itemize}

\section{The Pattern}

Altogether, the following approach tends to produce patterns that look like
this (again, simply guidance here, not substitute for your own
thinking/coding):

\begin{lstlisting}[style=netseclab,language=c]
/* Ordinary code */

// Lock mutex
pthread_mutex_lock(&lock);

// read/write concurrency state, maybe signal?

while(/* undesired condition */) {
  // wait on some signal with mutex LOCKED
  pthread_cond_wait(&cv, &lock);
}
// When at this point, I own the lock!

// read/write concurrency state, maybe signal?

// unlock mutex
pthread_mutex_unlock(&lock);

/* Code that has special rules, generally a critical section.
 * It is not uncommon for us to use sleep here to simulate something taking
 * place. */

// lock mutex
pthread_mutex_lock(&lock);

// read/write concurrency state, maybe signal?

// unlock mutex
pthread_mutex_unlock(&lock);

/* Ordinary code and other stuff, maybe return? */
\end{lstlisting}

\section{Some Pitfalls}

Here is a small list of typical pitfalls that we have seen students attempt to
do when they feel stuck on a problem. Generally, doing one of these is not a
good idea.

\begin{itemize}
  \item When reading the man pagees, you might discover new functions that
    might seem appealing, do not make a point to try out these new functions
    during an exam. Try those out on your free time.

    For example, some students, when running into a deadlock, discover that
    {\tt pthread\_mutex\_trylock} attempts to grab a lock, but does not block
    if it is not in an unclocked state. This is not the way to solve a
    deadlock!

  \item Swapping out {\tt pthread\_cond\_signal} for {\tt
    pthread\_cond\_broadcast} (and vice versa) in the hope that is magically
    solves a concurrency problem is not a good idea. You should be intentional
    about your signals/broadcasts.

  \item Do not try to dance with where your locks/unlocks should go. Changing
    the locations of your lock/unlock statements is not going to cut it. Go
    back to your paper and revisit your design.

  \item Make sure that each {\tt pthread\_cond\_wait} has at least one
    corresponding {\tt pthread\_cond\_signal} (or {\tt pthread\_cond\_broadcast}
    if necessary).

  \item Similarly, make sure that each {\tt pthread\_mutex\_lock} has exactly
    one corresponding {\tt pthread\_mutex\_unlock}.

    \noindent {\em Do not forget about the implicit lock/unlock statements in
    {\tt pthread\_cond\_wait}.}

  \item Be way of ``over-locking''. For example, doing something the following
    is generally a bad idea:

    \begin{lstlisting}[style=netseclab,language=c]
void *thread_fn(void *arg) {
  pthread_mutex_lock(&lock);
  while(/* some undesired condition */) {
    pthread_cond_wait(&cv, &lock);
  }

  // Code that has special rules, generally a critical section.

    pthread_cond_signal(&cv);
    pthread_mutex_unlock(&lock);
}
    \end{lstlisting}

    Note that in the code above, the {\bf entire} code block is guarded by a
    single lock, which renders the condition variables pretty much superfluous.
    Exacly one thread can be that area at a time, so your code will run
    completely serially, which defeats the purpose of using condition
    variables.

  \item If you need to count the number of waiting threads, be careful not to
    double count a thread that goes back to sleep. For example,

    \begin{lstlisting}[style=netseclab,language=c]
pthread_mutex_lock(&lock);
while(/* some undesired condition */) {
  num_waiting_threads++;
  pthread_cond_wait(&cv, &lock);
}
num_waiting_threads--;
    pthread_mutex_unlock(&lock);
    \end{lstlisting}

    This can be dangerous if multiple threads are awakened by a broadcast, but
    only a subset of them exit the loop. In that case, each thread that waits
    again will double itself as another waiting thread, and the number of
    waiting threads will never go back to 0.

    \noindent You can either do this:

    \begin{lstlisting}[style=netseclab,language=c]
pthread_mutex_lock(&lock);
num_waiting_threads++;
while(/* some undesired condition */) {
  pthread_cond_wait(&cv, &lock);
}
num_waiting_threads--;
pthread_mutex_unlock(&lock);
    \end{lstlisting}

    or,

    \begin{lstlisting}[style=netseclab,language=c]
pthread_mutex_lock(&lock);
while(/* some undesired condition */) {
  num_waiting_threads++;
  pthread_cond_wait(&cv, &lock);
  num_waiting_threads--;
}
pthread_mutex_unlock(&lock);
    \end{lstlisting}

\end{itemize}
  
\end{document}

