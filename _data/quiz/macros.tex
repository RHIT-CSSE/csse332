\newcommand{\tfidk}[1]{%
    \vspace{0.05in}
    \begin{oneparchoices}
        \ifnum#1=1
            \CorrectChoice T
        \else
            \choice T
        \fi
        \ifnum#1=2
            \CorrectChoice F
        \else
            \choice F
        \fi
        \choice IDK
    \end{oneparchoices}
    \vspace{0.05in}%
}

\newcommand{\Fix}[1]{\noindent\emph{\hl{TODO: #1}}}
\newcommand{\todo}[1]{\noindent\emph{\hl{TODO: #1}}}

\newcommand{\CodeIn}[1]{{\small \texttt{#1}}}
\newcommand{\CI}[1]{\CodeIn{#1}}
\newcommand{\bigtheta}[1]{\ensuremath{\Theta\left(#1\right)}}

\usepackage{xstring}
\newcommand{\PP}[1]{
    \smallskip
    \noindent{\bf \IfEndWith{#1}{.}{#1}{#1.}}
}

% easy paranthesis placement
\newcommand{\pr}[1]{\left(#1\right)}

% for eval macros
\newcommand{\DefMacro}[2]{\expandafter\newcommand\csname rmk-#1\endcsname{#2}}
\newcommand{\UseMacro}[1]{\csname rmk-#1\endcsname}

% for drawing the colored circles to mention the figure
%% Ported from Wajih's code
\newcommand*\BC[1]{%
    \begin{tikzpicture}[baseline=(C.base)]
        \node[draw,circle,fill=yyellow,inner sep=0.2pt](C) {\textcolor{black}{#1}};
    \end{tikzpicture}
}

%% lstlisting config

% Set the default code path
% \lstset{inputpath=code/}

% ----------
% THIS BELOW COPIED FROM vkuncak/doc/vmcai09/defs.tex
\definecolor{gray}{RGB}{211,211,211}
\definecolor{yyellow}{HTML}{FFFF99}
\definecolor{ghkeyword}{HTML}{d73a49}
\definecolor{ghnormal}{HTML}{032f62}
\newcommand{\jbasicstyle}{\scriptsize\sffamily}
\newcommand{\jnumberstyle}{\tiny}
\newcommand{\Hilight}{\makebox[0pt][l]{\color{gray}\rule[-3pt]{0.80\linewidth}
{9pt}}}
\lstdefinelanguage{pseudo}{
    morekeywords={if,else,return,map},
    keywordstyle=\bfseries,
    lineskip=-0.1em,
    numbers=none,
    numberstyle=\jnumberstyle,
    numbersep=4pt,
    basicstyle=\scriptsize,
    basicstyle=\jbasicstyle,
    breaklines=true,
    breakautoindent=true,
    commentstyle=\color{dkgreen},
    tabsize=2,
    columns=fullflexible,
    morecomment=*[l][\textsl]{//},
    mathescape=true,
}

\lstdefinelanguage{p4}{
    morekeywords={state, parser, packet_in, out, inout, metadata,
        standard_metadata_t,
        transition, select, default, table, key, action, default_action, apply, actions,
    size},
    morecomment=[l]{//},
    morecomment=[s]{/*}{*/},
    morestring=[b]",
    keywordstyle=\bfseries\color{ghkeyword},
    backgroundcolor=\color{white},
    basicstyle=\color{ghnormal},
}

\definecolor{dkgreen}{rgb}{0,.4,0}
\definecolor{gray}{rgb}{0.5,0.5,0.5}
\definecolor{mauve}{rgb}{0.58,0,0.82}
\definecolor{light-gray}{gray}{0.95} %the shade of grey that stack exchange uses
\newcommand{\cbasicstyle}{\scriptsize\sffamily}
\newcommand{\cnumberstyle}{\tiny}

% \lstset{frame=tb,
%     language=C,
%     backgroundcolor=\color{light-gray},
%     aboveskip=0mm,
%     belowskip=0mm,
%     % showstringspaces=false,
%     % columns=flexible,
%     numberstyle=\jnumberstyle,
%     basicstyle=\scriptsize,
%     basicstyle=\jbasicstyle,
%     numbers=none,
%     numberstyle=\tiny\color{gray},
%     keywordstyle=\color{blue},
%     commentstyle=\bfseries\color{dkgreen},
%     stringstyle=\color{mauve},
%     breaklines=true,
%     breakatwhitespace=true,
%     tabsize=2
% }

\SetKwProg{Fn}{Func}{}{}

\newcommand{\algorithmfootnote}[2][\footnotesize]{%
  \let\old@algocf@finish\@algocf@finish% Store algorithm finish macro
  \def\@algocf@finish{\old@algocf@finish% Update finish macro to insert "footnote"
    \leavevmode\rlap{\begin{minipage}{\linewidth}
    #1#2
    \end{minipage}}%
  }%
}

\newcolumntype{Y}{>{\centering\arraybackslash}X}

